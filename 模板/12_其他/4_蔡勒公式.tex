$w = \left(y+\left[\frac {y}{4}\right] + \left[\frac {c}{4}\right] - 2c + \left[\frac{26(m+1)} {10}\right] + d-1 \right) \mod 7$ \\
若要计算的日期是在1582年10月4日或之前,公式则为:\\
$w = \left(y+\left[\frac {y}{4}\right] + \left[\frac {c}{4}\right] - 2c + \left[\frac{26(m+1)} {10}\right] + d+2 \right) \mod 7$ \\
公式中的符号含义如下:\\
w:星期(计算所得的数值对应的星期:0-星期日;1-星期一;2-星期二;3-星期三;4-星期四;5-星期五;6-星期六)\\
c:年份前两位数\\
y:年份后两位数\\
m:月(m的取值范围为3至14,即在蔡勒公式中,某年的1、2月要看作上一年的13、14月来计算,比如2003年1月1日要看作2002年的13月1日来计算)\\
d:日\\
(因罗马教宗额我略十三世颁布新历法(公历),把1582年10月4日的后一天改为1582年10月15日)\\